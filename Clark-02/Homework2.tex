\documentclass[12pt]{article}
\usepackage{amsmath, amssymb, amsfonts, amsthm}
\usepackage{fullpage}
\usepackage{setspace}

\title{CMPT 404\\Homework 2}
\author{Daniel Clark}
\date{\today}

\begin{document}
\maketitle

\begin{description} 
\begin{doublespace}


\item[2.1] In Equation (2.1), set $\delta = 0.03$ and let
\[
    \epsilon(M, N, \delta) = \sqrt{\frac{1}{2N} ln \frac{2M}{\delta}}
\]
\item[2.1 (a)] For M = 1, how many examples do we need to make $\epsilon \leq 0.05$?

    First we will solve the equation to give $N$ as a function of $M$.
    
    \begin{align*}
        \epsilon &= \sqrt{\frac{1}{2N} ln \frac{2M}{\delta}}\\
        \epsilon^2 &= \frac{1}{2N} ln \frac{2M}{\delta}\\
        N &= \frac{1}{2\epsilon} ln \frac{2M}{\delta}\\
    \end{align*}
    
    Knowing the values of $\epsilon$ and $\delta$ we can get that 
\[
    N(M) = 10 ln \frac{200M}{3}
\]

    Plugging in $M = 1$ we find that $N(1) \approx 41.997$ so we need $42$ samples.
    
\item[2.1 (b)] $For M = 100$, how many examples do we need to make $\epsilon \leq 0.05$?

    Using the formula derived in part (a) we find that $N(100) \approx 88.049$ so we need $89$ samples.

\item[2.1 (c)] For $M = 10000$, how many examples do we need to make $\epsilon \leq 0.05$?

    Using the formula derived in part (a) we find that $N(10000) \approx 134.100$ so we need $135$ samples.


\item[2.11] Suppose $m_H(N) = N+1$, so $d_{vc} = 1$. You have $100$ training examples. Use the generalization bound to give a bound for $E_{out}$ with confidence $90\%$. Repeat for $N = 10000$.

Using equation (2.12), we can give a bound for $E_{out}$ in terms of $E_{in}$.
\begin{align*}
    E_{out}(g) &\leq E_{in}(g) + \sqrt{\frac{8}{N} ln \frac{4m_H(2N)}{\delta}}\\
    E_{out}(g) &\leq E_{in}(g) + \sqrt{\frac{8}{100} ln \frac{4m_H(200)}{0.1}}\\
    E_{out}(g) &\leq E_{in}(g) + \sqrt{\frac{2}{25} ln 8040}\\
    E_{out}(g) &\leq E_{in}(g) + 0.8482
\end{align*}
This means that the real error will be within $.8482$ of the observed error with a probability of $90\%$. When this is repeated with $N = 10000$, we find
\begin{align*}
    E_{out}(g) &\leq E_{in}(g) + \sqrt{\frac{8}{10000} ln \frac{4m_H(20000)}{0.1}}\\
    E_{out}(g) &\leq E_{in}(g) + \sqrt{\frac{1}{1250} ln 800040}\\
    E_{out}(g) &\leq E_{in}(g) + 0.1043
\end{align*}
This means that the real error will be within $0.1043$ of the observed error with a probability of $90\%$.


\item[2.12] For an $H$ with $d_{vc} = 10$, what sample size do you need (as prescribed by the generalization bound) to have a $95\%$ confidence that your generalization error is a most $0.05$?

Since we are working within a $95\%$ confidence interval, we know that $\delta = 0.05$. Additionally, out generalization error is $\epsilon = 0.05$. Using equation (2.13) we get that
\begin{align*}
    N &\geq \frac{8}{\epsilon^2} ln \frac{4((2N)^{d_{vc}} + 1)}{\delta}\\
    N &\geq \frac{8}{0.05^2} ln \frac{4((2N)^{10} + 1)}{0.05}\\
    N &\geq 3200 ln 80((2N)^{10} + 1)
\end{align*}
Starting with an initial guess of $N=10000$, we iterate until $N$ converges on $N = 452956.895$ so we need $452957$ samples.


Problem 3.1 follows

\end{doublespace}
\end{description}
\end{document}

